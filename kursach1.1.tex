\documentclass {beamer}	
\usepackage{amsmath}
\usepackage{amsfonts}
\usepackage{amssymb}
\usepackage[T2A]{fontenc}
\usepackage[UTF8]{inputenc}
\usepackage[english,russian]{babel}
\usetheme{Berlin}
\title{Курсовая работа\\ "Равновесия в моделях экономики"}
\author {Студент III курса ПМ \\ Бондаренко Алексей}
\date{}


\begin{document}

\begin{frame}
\maketitle
\end{frame}

\begin{frame}
\section{Дуополия Курно}
Функция спроса $p=a-bQ$, где $Q=q_1+q_2$, и $a>0,b>0.$ $$\pi_i=TR_i(q_i,q_j)-TC_i(q_i)\to \max_{q_i} ,i\ne j$$
\begin{equation} \label{qi}  q_i = \frac{a-c_i}{2b}-\frac {q_j}{2}.  \end{equation}
При одинаковых функциях издержки объем в точке равновесия равен: $$q_1 =q_2 = \frac {a-c}{3b}, \qquad\qquad Q=\frac{2}{3}\frac{a-c}{b}.$$
\end{frame}

\end{document}