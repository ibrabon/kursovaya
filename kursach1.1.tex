\documentclass {beamer}	
\usepackage{amsmath}
\usepackage{amsfonts}
\usepackage{amssymb}
\usepackage[T2A]{fontenc}
\usepackage[UTF8]{inputenc}
\usepackage[english,russian]{babel}
\usetheme{PaloAlto}
\title{Курсовая работа\\ "Равновесия в моделях экономики"}
\author { Бондаренко Алексей}
\date{}


\begin{document}
%%\sound{loop}{Radiohead - Creep.mp3}
\begin{frame}
\maketitle
\end{frame}

\begin{frame}
\section{Дуополия Курно}
\frametitle{Дуополия Курно}
Функция спроса $p=a-bQ$, где $Q=q_1+q_2$, и $a>0,b>0.$ $$\pi_i=TR_i(q_i,q_j)-TC_i(q_i)\to \max_{q_i} ,i\ne j$$
\begin{equation} \label{qi}  q_i = \frac{a-c_i}{2b}-\frac {q_j}{2}.  \end{equation}
При одинаковых функциях издержки объем в точке равновесия равен: $$q_1 =q_2 = \frac {a-c}{3b}, \qquad\qquad Q=\frac{2}{3}\frac{a-c}{b}.$$
\end{frame}

\begin{frame}
\section{Дуополия Штакельберга}
\frametitle{Дуополия Штакельберга}
Функция цены:\qquad\quad $p=a-bQ$ \\Функция прибыли: $${\pi_1 = p(q_1+q_2)*q_1-c_1 q_1 \atop \pi_2 = p(q_1+q_2)*q_2-c_2 q_2 }$$ Оптимальный выпуск при одинаковых функциях издержек:$$q^*_1=\frac{a-c}{2b} \quad q^*_2=\frac{a-c}{4b}$$\newpage Общий объем: $$Q_S=\frac{3}{4}\frac{a-c}{b}>Q_K=\frac{2}{3}\frac{a-c}{b} $$ Цена:$$ p_S=\frac{1}{4}a+\frac{3}{4}c<p_K=\frac{1}{3}a+\frac{2}{3}c$$\\*
\end{frame}

\begin{frame}
\section{Модель ''Борьба за лидерство''}
\frametitle{Борьба за лидерство}
Развитие моедли Штакельберга. Модель предполагает, что дуополисты максимизируют прибыль при условии, что конкуренты реагируют на действитя друг друга в соответствии со своими линиями реакции Курно (\ref{qi}). Максимизировав прибыли: $$q_1^* = q_2^* = \frac{2(a-c)}{5b} $$ Получаем оптимальные объем и цену: $$ Q^*=q_1^*+q_2^* = \frac{4(a-c)}{5b} \qquad \qquad p^*=\frac{a+4c}{5}$$
\end{frame}
\section{Дуополия Бертрана}
\begin{frame}

\frametitle{Дуополия Бертрана}
Функция рыночного спроса: $Q = \frac{a}{b}- \frac{1}{b}p$\\
В модели приняты следующие предположения: {\begin{itemize}\item \pauseФирмы ведут себя не кооперативно; 
\item \pauseФункция спроса линейна; 
\item \pauseФирмы конкурируют, устанавливая цену на свою продукцию, и выбирают ее независимо и одновременно; 
\item \pauseМодель статична\item \pause $$p=\frac{1}{5}a+\frac{4}{5}c\qquad q_1=\left\{ \begin{aligned} &Q, p_1<p_2 \\ &Q/2, p_1=p_2 \\& 0, p_1>p_2 \end{aligned}\right.$$
\end{itemize}}

\end{frame}

\begin{frame} 
\section{Модель Эджворта}
\frametitle{Модель Эджворта}
Модель Эджворта являет одно из решений парадокса Бертрана.
$$q_1  \le K_1, q_2 \le K_2 \qquad K_1 \le K_2 < (a-c) / b.$$
{\bf Поведение на рынке:\\}Модель Эджворта  не предполагает никакого статистического равновесия. Между фирмами будет бесконечная ценовая война, в которой падение цен чередуется с их всплесками.

\end{frame}

\begin{frame}
\section{Задача №1}
\frametitle{Задача №1}
Первая фирма производит одну единицу продукции, затрачивая $30$ единиц труда и $30$ единиц капитала. Вторая фирма производит одну единицу продукции, затрачивая $30$ единиц труда и $60$ единиц капитала. Цена единицы труда равна $w$, цена
	единицы капитала равна $r$. \begin{enumerate}
	\itemВычислим параметры равновесия Курно
	\itemПокажем, что прибыль второй фирмы не зависит от цены капитала. Рассмотрим, как это влияет на конкурентоспособность фирм в отрасли.
\end{enumerate}
\begin{block}
{$$ P=90-Q \qquad \qquad Q=q_1+q_2$$}
Параметры равновесия:$q_1=30-10w\quad q_2=30-10w-30r$\\
Прибыль второго дуополиста: $\pi_2=(30-10w)^2$
\end{block}
\end{frame}

\begin{frame}
\section{Задача №2}
\frametitle{Задача №2}
\begin{enumerate}
\item Вычислим параметры равновесия Курно. 
\item Покажем: если две из трех фирм сольются, превратив отрасль в дуополию, то прибыль вновь образовавшейся фирмы станет меньше, чем совокупная прибыль двух фирм, решивших создать одну
\itemРассмотрим, что произойдет с параметрами равновесия, если сольются все три фирмы 
\end{enumerate}
\begin{block}
{$$p=120-Q \quad Q=q_1+q_2+q_3 \quad MC=0$$}
\begin{enumerate}
\item$q_1=q_2=q_3=30\quad\pi_1=\pi_2=\pi_3=900$
\itemПервая и вторая фирма образуют четвертую. \\$q_3=q_4\quad \pi_3=\pi_4=1600 \Longrightarrow \pi_1 =\pi_2=800$
\item  $q=60 \pi=3600 \Longrightarrow \pi_1=\pi_2=\pi_3=1200$
\end{enumerate}
\end{block}
\end{frame}

\begin{frame}
\section{Задача №3}
\frametitle{Задача №3}
Спрос задается функцией $Q =20-0.5p $ , рассчитать равновесие по Курно и по Бертрану, когда$TC_i=10*q_i$. Какие будут изменения в равновесиях, если издержки одной из фирм возрастут:$TC_1=10*q_1\quad TC_2=16*q_2$
\begin{block}
{Результаты}
\begin{enumerate}
\item По Курно: $q=50\qquad \pi=50$\\ По Бертрану:\qquad $\pi=0$
\item По Курно: $\left\{ \begin{aligned} &q_1=6\\&q_2=3\end{aligned} \right. \left\{ \begin{aligned} &\pi_1=72 \\&\pi_2=18\end{aligned} \right.$\\ По Бертрану: $q=25 \quad\pi=112.5$  
\end{enumerate}
\end{block}
\end{frame}

\end{document}