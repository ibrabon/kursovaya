\documentclass {beamer}	
\usepackage{amsmath}
\usepackage{amsfonts}
\usepackage{amssymb}
\usepackage[T2A]{fontenc}
\usepackage[UTF8]{inputenc}
\usepackage[english,russian]{babel}
\usetheme{PaloAlto}
\title{Курсовая работа\\ "Равновесия в моделях экономики"}
\author {Студент III курса ПМ \\ Бондаренко Алексей}
\date{}


\begin{document}

\begin{frame}
\maketitle
\end{frame}

\begin{frame}
\section{Дуополия Курно}
\frametitle{Дуополия Курно}
Функция спроса $p=a-bQ$, где $Q=q_1+q_2$, и $a>0,b>0.$ $$\pi_i=TR_i(q_i,q_j)-TC_i(q_i)\to \max_{q_i} ,i\ne j$$
\begin{equation} \label{qi}  q_i = \frac{a-c_i}{2b}-\frac {q_j}{2}.  \end{equation}
При одинаковых функциях издержки объем в точке равновесия равен: $$q_1 =q_2 = \frac {a-c}{3b}, \qquad\qquad Q=\frac{2}{3}\frac{a-c}{b}.$$
\end{frame}

\begin{frame}
\section{Дуополия Штакельберга}
\frametitle{Дуополия Штакельберга}
Функция цены:\qquad\quad $p=a-bQ$ \\Функция прибыли: $${\pi_1 = p(q_1+q_2)*q_1-c_1 q_1 \atop \pi_2 = p(q_1+q_2)*q_2-c_2 q_2 }$$ Оптимальный выпуск при одинаковых функциях издержек:$$q^*_1=\frac{a-c}{2b} \quad q^*_2=\frac{a-c}{4b}$$\newpage Общий объем: $$Q_S=\frac{3}{4}\frac{a-c}{b}>Q_K=\frac{2}{3}\frac{a-c}{b} $$ Цена:$$ p_S=\frac{1}{4}a+\frac{3}{4}c<p_K=\frac{1}{3}a+\frac{2}{3}c$$\\*
\end{frame}

\begin{frame}
\section{Модель ''Борьба за лидерство''}
\frametitle{Борьба за лидерство}
Развитие моедли Штакельберга. Модель предполагает, что дуополисты максимизируют прибыль при условии, что конкуренты реагируют на действитя друг друга в соответствии со своими линиями реакции Курно (\ref{qi}). Максимизировав прибыли: $$q_1^* = q_2^* = \frac{2(a-c)}{5b} $$ Получаем оптимальные объем и цену: $$ Q^*=q_1^*+q_2^* = \frac{4(a-c)}{5b} \qquad \qquad p^*=\frac{a+4c}{5}$$
\end{frame}
\section{Дуополия Бертрана}
\begin{frame}

\frametitle{Дуополия Бертрана}
Функция рыночного спроса: $Q = \frac{a}{b}- \frac{1}{b}p$\\
В модели приняты следующие предположения: {\begin{itemize}\item \pauseФирмы ведут себя не кооперативно; 
\item \pauseФункция спроса линейна; 
\item \pauseФирмы конкурируют, устанавливая цену на свою продукцию, и выбирают ее независимо и одновременно; 
\item \pauseМодель статична\item \pause $$p=\frac{1}{5}a+\frac{4}{5}c\qquad q_1=\left\{ \begin{aligned} &Q, p_1<p_2 \\ &Q/2, p_1=p_2 \\& 0, p_1>p_2 \end{aligned}\right.$$
\end{itemize}}

\end{frame}

\end{document}